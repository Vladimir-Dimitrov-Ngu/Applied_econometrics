\documentclass[a4paper,12pt]{article}

%%% Работа с русским языком
\usepackage{cmap}					% поиск в PDF
\usepackage{mathtext} 				% русские буквы в фомулах
\usepackage[T2A]{fontenc}			% кодировка
\usepackage[utf8]{inputenc}			% кодировка исходного текста
\usepackage[english,russian]{babel}	% локализация и переносы
\usepackage{float}
%%% Дополнительная работа с математикой
\usepackage{amsfonts,amssymb,amsthm,mathtools} % AMS
\usepackage{amsmath}
\usepackage{icomma} % "Умная" запятая: $0,2$ --- число, $0, 2$ --- перечисление

%% Номера формул
%\mathtoolsset{showonlyrefs=true} % Показывать номера только у тех формул, на которые есть \eqref{} в тексте.

%% Шрифты
\usepackage{euscript}	 % Шрифт Евклид
\usepackage{mathrsfs} % Красивый матшрифт

%% Свои команды
\DeclareMathOperator{\sgn}{\mathop{sgn}}

%% Перенос знаков в формулах (по Львовскому)
\newcommand*{\hm}[1]{#1\nobreak\discretionary{}
	{\hbox{$\mathsurround=0pt #1$}}{}}

%%% Работа с картинками
\usepackage{graphicx}  % Для вставки рисунков
\graphicspath{{images/}{images2/}}  % папки с картинками
\setlength\fboxsep{3pt} % Отступ рамки \fbox{} от рисунка
\setlength\fboxrule{1pt} % Толщина линий рамки \fbox{}
\usepackage{wrapfig} % Обтекание рисунков и таблиц текстом

%%% Работа с таблицами
\usepackage{array,tabularx,tabulary,booktabs} % Дополнительная работа с таблицами
\usepackage{longtable}  % Длинные таблицы
\usepackage{multirow} % Слияние строк в таблице
\usepackage[unicode, pdftex]{hyperref}
\usepackage{xcolor}
\definecolor{urlcolor}{HTML}{0000ff}
\definecolor{linkcolor}{HTML}{799B03}
\hypersetup{pdfstartview=FitH,  linkcolor=linkcolor,urlcolor=urlcolor, colorlinks=true}
%%% Заголовок
\author{Владимир Димитров, Владислав Балабаев гр. 19701}
\title{Описание работы}
\date{\today}

\begin{document}
	\maketitle
	\newpage
	\section{Отчет по заданию}
	Первая часть задания направлена на создания новых переменных. В таблице 1 представлены новые переменные. 

	\begin{figure}[H]
		\centering
		\includegraphics[width=0.9\linewidth]{C:/Users/user/DL_audio/Work/Отчет/Image/table}
		
		\label{fig:table}
	\end{figure}
	\begin{center}
		Таблица 1: новые переменные
	\end{center}
	
	В таблице 2 выведены описательный статистики.
	
	\begin{figure}[H]
		\centering
		\includegraphics[width=0.9\linewidth]{C:/Users/user/DL_audio/Work/Отчет/Image/table2}
		\label{fig:table2}
	\end{figure}
	\begin{center} Таблица 2: описательные статистики
	\end{center}
	Была оценена модель предложения труда для женщин. В качестве зависимой переменой выступали часы работы, а независимой переменной были: логарифм почасовой зарплаты и нетрудового дохода, возраст, возраст в квадрате, фиктивные переменные для образования, семейное положение и наличие детей до 18 лет. Результаты представлены в таблице 3. На уровне значимости 5 процентов, значимыми фактами оказались: логарифм почасовой зарплаты, получение высшего образования и семейное положение.
	\begin{figure}[H]
		\centering
		\includegraphics[width=1\linewidth]{C:/Users/user/DL_audio/Work/Отчет/Image/tabl3}
		\label{fig:tabl3}
	\end{figure}
	\begin{center} Таблица 3: Наивная оценка регрессии
	\end{center}
	В результате последовательного удаления незначимых факторов, была получена регрессионная модель, где все переменные значимы на 5\%-ом уровне. На уровне значимости 5 процентов модель оказалась значимой.
	
	\begin{figure}[H]
		\centering
		\includegraphics[width=1\linewidth]{C:/Users/user/DL_audio/Work/Отчет/Image/tabl4}
		\label{fig:tabl4}
	\end{figure}
	\begin{center} Таблица 4: Оценка регрессии после удаление признаков
	\end{center}
	Рассмотрим модель, которая является устойчивой к гетероскедастичности.
	\begin{figure}[H]
		\centering
		\includegraphics[width=1\linewidth]{C:/Users/user/DL_audio/Work/Отчет/Image/tabl5}
		\label{fig:tabl5}
	\end{figure}
	\begin{center} Таблица 5: Наивная робустная модель
\end{center}
	 	Многие факторы являются незначимыми. Проведем процедуру исключения незначимых факторов.
	 	
	 	\begin{figure}[H]
	 		\centering
	 		\includegraphics[width=1\linewidth]{C:/Users/user/DL_audio/Work/Отчет/Image/tabl6}
	 		\label{fig:tabl6}
	 	\end{figure}
 		\begin{center} Таблица 6: Робустная модель с значимыми признаками
 	\end{center}
	 	Интерпретация полученных моделей: 
	 	\begin{itemize}
	 		\item наличие высшего образования у женщины уменьшает продолжительность ее рабочего дня на 54 минуты.
	 		\item продолжительность рабочего дня женщины в среднем составляет 10 часов.
	 		\item при увеличении нетрудового дохода на 1\% женщины работают на 0,0003 часа в день меньше.
	 		
	 	\end{itemize}
 	
 Оцененный коэффициент эластичности равен -0,023. Следовательно, при росте платы на 1\% количество рабочих часов в день снижается на 0,023\%.
 
 Для того чтобы оценки были состоятельны, нам необходимо учесть тех людей, которые не работают. Построим регрессионную модель, которая предсказывает значение заработной платы для неработающего населения.
 
 \begin{figure}[H]
 	\centering
 	\includegraphics[width=1\linewidth]{C:/Users/user/DL_audio/Work/Отчет/Image/tabl7}
 	\label{fig:tabl7}
 \end{figure}
\begin{center}
	Таблица 7: Модель для неработающего населения
\end{center}

Были созданы новые переменные, lhourwage\_p (предсказанные значения заработной платы на основе работающей выборки), hours\_p (0 для неработающего населения). В таблице 8 оценена модель устойчивая к гетероскедастичности.

\begin{figure}[H]
	\centering
	\includegraphics[width=1\linewidth]{C:/Users/user/DL_audio/Work/Отчет/Image/tabl8}
	\label{fig:tabl8}
\end{figure}
\begin{center}
	Таблица 8: Предложения труда для женщин на полной выборке
\end{center}
Интерпретация полученных результатов:
\begin{itemize}
\item	наличие высшего образования у женщины уменьшает продолжительность ее рабочего дня на 38 минут
\item если женщина состоит в зарегистрированном браке, то рабочее время сокращается в среднем 24 минуты
\item c возрастом женщины работают в среднем на полчаса в день больше
\end{itemize}
Коэффициент эластичности предложения труда женщин по заработной плате в полной выборке составил -0,01. Интерпретация такова: при росте почасовой заработной платы на 1\% продолжительность рабочего дня снижается на 0,01\%.

Как можно понять по данной работе, для более правдивых результатов необходимо использовать большую часть данных. 

\newpage

\Huge
\href{}{КОД}
 
\end{document}
	